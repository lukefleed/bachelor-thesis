% --- SLIDE 1: Titolo (Non mostrata, ma è la tua slide di titolo) ---

% --- SLIDE 2: The Subset Membership Problem ---
\begin{frame}{The Subset Membership Problem}
    \framesubtitle{Querying Collections Efficiently and Compactly}

    Consider a large universe of items $U = \{1, \dots, n\}$, and a specific subset $S \subseteq U$ of $m$ items.

    \begin{alertblock}{Core Task \& Desired Properties}
        \begin{itemize}
            \item Quickly answer: "Is item $x$ in $S$?" (\textbf{Membership Query})
            \item Store $S$ using minimal space (\textbf{Compact Representation})
        \end{itemize}
    \end{alertblock}
    \pause % Pausa prima del ponte a Shannon

    To understand "minimal space", we turn to Information Theory.
    \textit{What is the least number of bits needed to uniquely identify $S$?}
    \begin{definition}[Shannon Entropy $H(X)$]
        The average uncertainty, or information content, per symbol of source $X$:
        \[ H(X) = E_{P_X}[-\log_2 P_X(x)] = -\sum_{x \in \mathcal{X}} P_X(x) \log_2 P_X(x) \quad [\text{bits/symbol}] \]
    \end{definition}
\end{frame}

% --- SLIDE 3: Information-Theoretic Limits for Subsets ---
\begin{frame}{Information-Theoretic Limits for Subsets}
    \framesubtitle{From General Entropy to Specific Subsets}
    We usually don't know the true source $P_X$. We only have the data sequence $S$.
    \pause
    \begin{definition}[Zero-Order Empirical Entropy $\mathcal{H}_0(S)$]
        Information content of sequence $S$ based on its symbol counts ($n_s$):
        \[ \mathcal{H}_0(S) = \sum_{s \in \Sigma} \frac{n_s}{n} \log_2 \frac{n}{n_s} \quad [\text{bits/symbol}] \]
        \vspace{-0.2em}
        Uses observed frequencies $\frac{n_s}{n}$ instead of unknown $P_X(x)$.
    \end{definition}

    \pause % Pausa 2
    For our subset $S$ of $m$ items from $n$:
    \begin{itemize}
        \item There are $\binom{n}{m}$ such distinct subsets.
        \item To uniquely identify one, we need at least $\lceil \log_2 \binom{n}{m} \rceil$ bits. This is the \alert{information content} of specifying the subset.
    \end{itemize}

\end{frame}

% --- SLIDE 4: Bitvectors: Querying the Implicit Representation ---
\begin{frame}{Bitvectors: Querying the Implicit Representation}
    \framesubtitle{From Compact Storage to Element Access}

    We can represent our subset $S$ as a \textbf{bitvector} $B[1..n]$ ($B[i]=1 \iff i \in S$). We are \textbf{encoding the choice of $m$ positions for the '1's}, allowing us to store $B$ using $\approx \lceil \log_2 \binom{n}{m} \rceil$ bits. \textit{This means $B$ is not stored as an explicit array of $n$ bits.}

    % Immagine: gestita con \only per cambiare con le pause del testo
    \begin{figure}[htbp]
        \centering
        \only<1-3>{\includegraphics[width=0.95\textwidth]{assets/rank_select_1.pdf}} % Stati 1, 2, 3
        \only<4-5>{\includegraphics[width=0.95\textwidth]{assets/rank_select_2.pdf}}   % Stati 4, 5
        \only<6->{\includegraphics[width=0.95\textwidth]{assets/rank_select_3.pdf}}  % Stato 6 in poi
        \vspace{-0.5em} % Leggero aggiustamento verticale se necessario dopo l'immagine
    \end{figure}

    \pause % STATO 1 -> STATO 2 (appare la domanda)

    \uncover<2->{If $B$ is not explicit, how do we access $B[i]$? We use two foundational queries:}

    \pause % STATO 2 -> STATO 3 (appare def. rank)

    \begin{columns}[T, totalwidth=\textwidth] % Le colonne esistono, ma il contenuto appare in fasi
        \begin{column}{0.52\textwidth}
            \uncover<7->{ % Il blocco "Access Queries" appare per ultimo (STATO 7)
                \begin{block}{\textsf{Access} Queries}
                    $B[i] = 1 \iff \textsf{rank}_1(B, i) > \textsf{rank}_1(B, i-1)$
                \end{block}
            }
        \end{column}
        \begin{column}{0.45\textwidth}
            \begin{itemize}
                \item<3-> \textbf{\textsf{rank}}$_b(B, i)$: How many bits $b$ are in the prefix $B[1..i]$?
                    \pause % STATO 3 -> STATO 4 (immagine rank_select_2)
                \item<5-> \textbf{\textsf{select}}$_b(B, j)$: What is the position of the $j$-th occurrence of bit $b$?
                    \pause % STATO 5 -> STATO 6 (immagine rank_select_3)
            \end{itemize}
        \end{column}
    \end{columns}
\end{frame}

% --- SLIDE 5: RRR Structure: The Bitvector Solution ---
\begin{frame}{RRR Structure: The Bitvector Solution}
    \framesubtitle{$n\mathcal{H}_0(B)$ Space \& $O(1)$ Queries}
    \begin{block}{Succinct Data Structure for Bitvectors}
        \begin{itemize}
            \item \textbf{Goal:} Support $\textsf{rank}$ and $\textsf{select}$ in $O(1)$ time.
            \item \textbf{Space:} Close to the information-theoretic minimum for the bitvector.
        \end{itemize}
    \end{block}
    \pause
    \begin{theorem}[RRR Structure]
        A bitvector $B[1..n]$ with $m$ set bits can be represented using
        \[ B(n, m) + o(n) + O(\log \log n) \quad \text{bits}, \]
        where $B(n, m) = \lceil \log_2 \binom{n}{m} \rceil$, while supporting \textsf{rank} and \textsf{select} queries in $O(1)$ time.
    \end{theorem}
    \pause
    \begin{alertblock}{A Cornerstone Result}
        RRR shows that \textbf{optimal space} \emph{and} \textbf{efficient queries} are possible for subsets.
    \end{alertblock}
\end{frame}

% --- SLIDE 6: Why Succinct Data Structures? ---
\begin{frame}{Why Succinct Data Structures?}
    \framesubtitle{The General Challenge \& Goal}

    RRR is a specific solution to a general problem:
    \begin{block}{Massive Data \& Auxiliary Structures Overhead}
        Modern datasets (Science, Web, AI...) are enormous. Complex analysis demands data in RAM, but auxiliary structures (indexes, trees) needed for queries often \textbf{occupy more space than the data itself}.
        $\implies$ Fitting everything in RAM is a major bottleneck.
    \end{block}
    \pause
    \begin{columns}[T]
        \begin{column}<2->{0.45\textwidth}
            \textbf{Classic Trade-off:}
            \begin{itemize}
                \item \textit{Compression:} Minimal space, but slow/no direct queries.
                \item \textit{Traditional Data Structures:} Fast queries, but large space overhead.
            \end{itemize}
        \end{column}
        \begin{column}<3->{0.55\textwidth}
            \begin{alertblock}{The Succinct Goal: Best of Both Worlds}
                Can we achieve \textbf{both}?
                \begin{itemize}
                    \item Space \alert{near information-theoretic minimum}.
                    \item Efficient queries \alert{directly} on compact data.
                \end{itemize}
            \end{alertblock}
        \end{column}
    \end{columns}
\end{frame}
