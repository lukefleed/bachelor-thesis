% Chapter 2

\chapter{Preliminaries} % Chapter title

\label{ch:Chapter2} % For referencing the chapter elsewhere, use \autoref{ch:Chapter2}

% -----------------------------------------------------------------------------------

\section{Explain what we want to do}

\lipsum[1-2]

% -----------------------------------------------------------------------------------

\section{Suffix Array}

Provides fast patter search operations

\subsection{(?) maybe suffix tree}

\lipsum[3-4]

% -----------------------------------------------------------------------------------

\section{Burrows Wheels Transform}

\lipsum[5-6]

\subsection{Suffix Array + BTW = FM-Index}

\lipsum[7-8]

% -----------------------------------------------------------------------------------

\section{Binary Wavelet Trees}

FM-Indexes use Backward Searches on the BWT to provide fast pattern matching, counting, and substring extraction
operations. Backward Search requires rank operations, which are best implemented using a Wavelet Tree. A WT encodes a string as a hierarchy of bit vectors, which it uses to answer general rank queries using $\log \sigma$ binary rank queries.

\subsection{A few pages about Wavelet trees}

\lipsum[9-10]

\subsection{WT and RRR}

Binary rank queries can be answered in O(1) time when the bit vector is stored as a RRR sequence, which utilize a global table of pre-calculated ranks. The RRR data structure also offers compression of the Wavelet Tree.

\subsection{(?) maybe a few words about multiary WT}

\lipsum[1-2]

% -----------------------------------------------------------------------------------

\section{Subset Wavelet Trees ????}
