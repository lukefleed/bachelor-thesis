% Chapter 2

\chapter{Compression Concepts and Techniques} % Chapter title

\label{ch:Chapter2} % For referencing the chapter elsewhere, use \autoref{ch:Chapter2}

TODO: Some introduction about the concept and idea of entropy, taken from \cite{Shannon1948,navarro2016compact,han2002mathematics}, Talk about worst case entropy from \cite{navarro2016compact}

\section{Shannon Entropy} \label{sec:shannon_entropy}

Let's introduce the concept of entropy as a measure of uncertainty of a random variable. A deeper explanation can be found in \cite{han2002mathematics,navarro2016compact,ElementsofInformationTheory}

\begin{definition}[Entropy of a Random Variable]\label{def:entropy}
    Let $X$ be a random variable taking values in a finite alphabet $\mathcal{X}$ with the probabilistic distribution $P_X(x)= \text{Pr}\{X=x\}~(x\in\mathcal{X})$. Then, the entropy of $X$ is defined as
    \begin{equation}
        H(X) = H(P_X) \myeq E_{P_x} \{-\log P_X(x)\} = -\sum_{x\in\mathcal{X}} P_X(x)\log P_X(x)
    \end{equation}
\end{definition}
\noindent Where $E_P$ denotes the expectation with respect to the probability distribution $P$. The $\log$ is taken to the base 2 and the entropy is expressed in bits. It is then clear that the entropy of a discrete random variable will always be nonnegative\footnote{The entropy is null if and only if $X = c$, where $c$ is a costant with probability one}.

\begin{example}[Toss of a fair coin]
    Let $X$ be a random variable representing the outcome of a toss of a fair coin. The probability distribution of $X$ is $P_X(0) = P_X(1) = \frac{1}{2}$. The entropy of $X$ is
    \begin{equation}
        H(X) = -\frac{1}{2}\log\frac{1}{2} - \frac{1}{2}\log\frac{1}{2} = 1
    \end{equation}
    This means that the toss of a fair coin has an entropy of 1 bit.
\end{example}

\begin{remark}
    Due to historical reasons, we are abusing the notation and using $H(X)$ to denote the entropy of the random variable $X$. It's important to note that this is not a function of the random variable: it's a functional of the distribution of $X$. It does not depend on the actual values taken by the random variable, but only on the probabilities of these values.
\end{remark}
The concept of entropy, introduced in definition \ref{def:entropy}, helps us quantify the randomness or uncertainty associated with a random variable. It essentially reflects the average amount of information needed to identify a specific value drawn from that variable. Intuitively, we can think of entropy as the average number of digits required to express a sampled value.

However, for continuous random variables (variables with an infinite number of possible values), expressing a single value with perfect accuracy requires an infinite number of digits. This is because any finite number of digits will only represent a range of possible values, not a single precise value. As a result, when we apply the definition of entropy to a continuous variable by dividing its range into increasingly smaller intervals and taking the limit, the entropy diverges to infinity.


\subsection{Properties}
In the previous section \ref{sec:shannon_entropy}, we have introduced the entropy of a single random variable $X$. What if we have two random variables $X$ and $Y$? How can we measure the uncertainty of the pair $(X,Y)$? This is where the concept of joint entropy comes into play. The idea is to consider $(X,Y)$ as a single vector-valued random variable and compute its entropy. This is the joint entropy of $X$ and $Y$.

\begin{definition}[Joint Entropy]\label{def:joint_entropy}
    Let $(X,Y)$ be a pair of discrete random variables $(X,Y)$ with a joint distribution $P_{XY}(x,y) = \text{Pr}\{X=x,Y=y\}$. The joint entropy of $(X,Y)$ is defined as
    \begin{equation}\label{eq:joint_entropy}
        H(X,Y) = H(P_{XY}) = -\sum_{x\in\mathcal{X}}\sum_{y\in\mathcal{Y}} P_{XY}(x,y)\log P_{XY}(x,y)
    \end{equation}
\end{definition}
\noindent Which we can be extended to the joint entropy of $n$ random variables $(X_1,X_2,\ldots,X_n)$ as $H(X_1,\ldots, X_n)$. \vspace*{0.4cm}

\noindent We also define the conditional entropy of a random variable given another as the expected value of the entropies of the conditional distributions, averaged over the conditioning random variable. Given two random variables $X$ and $Y$, we can define $W(y|x)$, with $x \in \mathcal{X}$ and $y \in \mathcal{Y}$, as the conditional probability of $Y$ given $X$. The set $W$ of those conditional probabilities is called \emph{channel} with \emph{input alphabet} $\mathcal{X}$ and \emph{output alphabet} $\mathcal{Y}$.

\begin{definition}[Conditional Entropy]\label{def:conditional_entropy}
    Let $(X,Y)$ be a pair of discrete random variables with a joint distribution $P_{XY}(x,y) = \text{Pr}\{X=x,Y=y\}$. The conditional entropy of $Y$ given $X$ is defined as
    \begin{align}
        H(Y|X) &= H(W | P_X) \myeq \sum_x P_X(x)H(Y|x) \\
        &= \sum_{x \in \mathcal{X}} P_X(x) \Big\{ -\sum_{y \in \mathcal{Y}} W(y|x) \log W(y|x) \Big\} \\
        &= -\sum_{x\in\mathcal{X}}\sum_{y\in\mathcal{Y}} P_{XY}(x,y)\log W(y|x) \\
        &= E_{P_{XY}} \{ -\log W(Y|X) \}
    \end{align}
\end{definition}
Since entropy is always nonnegative, conditional entropy is likewise nonnegative; it has value zero if and only if $Y$ can be entirely determined from $X$ with certainty, meaning there exists a function $f(X)$ such that $Y = f(X)$ with probability one. \vspace*{0.4cm}

\noindent The intuitive coherence of the definitions of joint entropy and conditional entropy becomes apparent when considering that the entropy of two random variables is equal to the entropy of one of them plus the conditional entropy of the other. This relationship is formally demonstrated in the following theorem.


\begin{theorem}[Chain Rule]\label{thm:chain_rule}
        Let $(X,Y)$ be a pair of discrete random variables with a joint distribution $P_{XY}(x,y)$. Then, the joint entropy of $(X,Y)$ can be expressed as \marginpar{This is also known as additivity of entropy.}
    \begin{equation}
        H(X,Y) = H(X) + H(Y|X)
    \end{equation}
\end{theorem}
\begin{proof}
    From the definition of conditional entropy (\ref{def:conditional_entropy}), we have
    \begin{align*}
        H(X,Y) &= -\sum_{x,y} P_{XY}(x,y) \log W(y|x) \\
        &= -\sum_{x,y} P_{XY}(x,y) \log \frac{P_{XY}(x,y)}{P_X(x)} \\
        &= -\sum_{x,y} P_{XY}(x,y) \log P_{XY}(x,y) + \sum_{x,y} P_{X}(x) \log P_X(x) \\
        &= H(XY) + H(X)
    \end{align*}
    Where we used the relation
    \begin{equation}
        W(y|x) = \frac{P_{XY}(x,y)}{P_X(x)}
    \end{equation}
    When $P_X(x) \neq 0$.
\end{proof}

\begin{corollary}
    \begin{equation}
        H(X, Y|Z) = H(X|Z) + H(Y|X,Z)
    \end{equation}
\end{corollary}
\begin{proof}
    The proof is analogous to the proof of the chain rule.
\end{proof}

\begin{corollary}
    \begin{align}
        H(X_1, X_2, \ldots, X_n) &= H(X_1) + H(X_2|X_1) + H(X_3|X_1, X_2) \nonumber \\
        &+ \ldots + H(X_n|X_1, X_2, \ldots, X_{n-1})
    \end{align}
\end{corollary}
\begin{proof}
    We can apply the two-variable chain rule in repetition obtain the result.
\end{proof}

\subsection{Mutual Information}
Given two random variables $X$ and $Y$, the mutual information between them quantifies the reduction in uncertainty about one variable due to the knowledge of the other. It is defined as the difference between the entropy and the conditional entropy

\begin{definition}[Mutual Information]\label{def:mutual_information}
    Let $(X,Y)$ be a pair of discrete random variables with a joint distribution $P_{XY}(x,y)$. The mutual information between $X$ and $Y$ is defined as
    \begin{equation}
        I(X;Y) = H(X) - H(X|Y)
    \end{equation}
\end{definition}
\noindent Using the chain rule (\ref{thm:chain_rule}), we can rewrite it as
\begin{align}
    I(X;Y) &= H(X) - H(X|Y) \nonumber \\
    &= H(X) + H(Y) - H(X,Y) \\
    &= -\sum_x P_X(x)\log P_X(x) - \sum_y P_Y(y)\log P_Y(y) \nonumber \\
    & \quad + \sum_{x,y} P_{XY}(x,y)\log P_{XY}(x,y) \\
    &= \sum_{x,y} P_{XY}(x,y)\log \frac{P_{XY}(x,y)}{P_X(x)P_Y(y)} \\
    &= E_{P_{XY}} \left\{ \log \frac{P_{XY}(x,y)}{P_X(x)P_Y(y)} \right\}
\end{align}
It immediately that the mutual information is symmetric, $I(X;Y) = I(Y;X)$.

% \begin{tikzpicture}

%     % Define the circles
%     \def\radius{2cm}
%     \def\dist{3.5cm}
%     \coordinate (center1) at (0,0);
%     \coordinate (center2) at (\dist,0);

%     % Draw circles
%     \draw (center1) circle [radius=\radius] node [above] {$H(Y)$};
%     \draw (center2) circle [radius=\radius] node [above] {$H(X)$};

%     % Label intersection
%     \coordinate (intersection) at (\dist/2,0);
%     \node at (intersection) [below] {$I(X;Y)$};

% \end{tikzpicture}


\subsection{Fano's inequality}

Information theory serves as a cornerstone for understanding fundamental limits in data transmission and compression. It not only allows us to prove the existence of encoders achieving demonstrably good performance, but also establishes a theoretical barrier against surpassing this performance. The following theorem, known as Fano's inequality, provides a lower bound on the probability of error in guessing a random variable $X$ to it's conditional entropy $H(X|Y)$, where $Y$ is another random variable\footnote{We have seen in \ref{def:conditional_entropy} that the conditional entropy of $X$ given $Y$ is zero if and only if $X$ is a deterministic function of $Y$. Hence, we can estimate $X$ from $Y$ with zero error if and only if $H(X|Y) = 0$.}.

\begin{theorem}[Fano's Inequality]\label{thm:fano_inequality}
    Let $X$ and $Y$ be two discrete random variables with $X$ taking values in some discrete alphabet $\mathcal{X}$, we have
    \begin{equation}
        H(X|Y) \leq \text{Pr}\{X \neq Y\} \log (|\mathcal{X}|-1) + h(\text{Pr}\{X \neq Y\})
    \end{equation}
    where $h(p) = -p\log p - (1-p)\log(1-p)$ is the binary entropy function.
\end{theorem}
\begin{proof}
    Let $Z$ be a random variable defined as follows:
    \begin{equation}\label{eq:random_variable_Z}
        Z = \begin{cases}
            1 & \text{if } X \neq Y \\
            0 & \text{if } X = Y
        \end{cases}
    \end{equation}
    We can then write
    \begin{align} \label{eq:entropy_decomposition}
        H(X|Y) &= H(X|Y) + H(Z|XY) = H(XZ|Y) \nonumber \\
        &= H(X|YZ) + H(Z|Y) \nonumber \\
        &\leq H(X|YZ) + H(Z)
    \end{align}
    The last inequality follows from the fact that conditioning reduces entropy. We can then write
    \begin{equation} \label{eq:entropy_decomposition2}
        H(Z) = h(\text{Pr}\{X \neq Y\})
    \end{equation}
    Since $\forall y \in \mathcal{Y}$, we can write
    \begin{equation}
        H(X | Y =y, Z =0) =0
    \end{equation}
    and
    \begin{equation}
        H(X| Y = y, Z = 1) \leq \log(|\mathcal{X}|-1)
    \end{equation}
    Combining these results, we have
    \begin{equation} \label{eq:entropy_decomposition3}
        H(X | YZ) \leq \text{Pr}\{X \neq Y\} \log (|\mathcal{X}|-1)
    \end{equation}
    From equations \ref{eq:entropy_decomposition}, \ref{eq:entropy_decomposition2} and \ref{eq:entropy_decomposition3}, we have Fano's inequality.
\end{proof}

\clearpage
\section{Empirical Entropy}
Introduction to empirical entropy, from \cite{han2002mathematics} in section 2.6 and in \cite{navarro2016compact} at section 2.3
\subsection{Bit Sequences}
\subsection{Sequences of Symbols}

Section 2.4 from \cite{navarro2016compact}

\clearpage
\section{Source and Code}

We enrich the understanding of entropy by establishing its core role in setting the fundamental limit for information compression. This process involves condensing data by assigning shorter descriptions to more frequent outcomes and longer descriptions to less frequent ones. For example, Morse code uses a single dot to represent the most common symbol. Within this chapter, we ascertain the minimum average description length for a random variable. \cite{ElementsofInformationTheory}

\subsection{Codes}

A source characterized by a random process generates symbols (letters) from a specific alphabet at each time step. The objective is to transform this output sequence into a more concise representation. This data reduction technique, known as \emph{source coding} or \emph{data compression}, utilizes a code to represent the original symbols more efficiently. The device that performs this transformation is termed an \emph{encoder}, and the process itself is referred to as \emph{encoding}. \cite{han2002mathematics}

\begin{definition}[Source Code]\label{def:code}
    A source code for a random variable $X$ is a mapping from the set of possible outcomes of $X$, called $\mathcal{X}$, to $\mathcal{D}^*$, the set of all finite-length strings of symbols from a $\mathcal{D}$-ary alphabet. Let $C(X)$ denote the codeword assigned to $x$ and let $l(x)$ denote length of $C(x)$
\end{definition}

\begin{definition}[Expected length]\label{def:expected_length}
    The expected length $L(C)$ of a source code $C$ for a random variable $X$ with probability mass function $P_X(x)$ is defined as
    \begin{equation}
        L(C) = \sum_{x\in\mathcal{X}} P_X(x)l(x)
    \end{equation}
    where $l(x)$ is the length of the codeword assigned to $x$.
\end{definition}
Let's assume from now for simplicity that the $\mathcal{D}$-ary alphabet is $\mathcal{D} = \{0, 1, \ldots, D-1\}$.

\begin{example}
    Let's consider a source code for a random variable $X$ with $\mathcal{X} = \{a, b, c, d\}$ and $P_X(a) = 0.5$, $P_X(b) = 0.25$, $P_X(c) = 0.125$ and $P_X(d) = 0.125$. The code is defined as
    \begin{align*}
        C(a) &= 0 \\
        C(b) &= 10 \\
        C(c) &= 110 \\
        C(d) &= 111
    \end{align*}
    The entropy of $X$ is
    \begin{equation*}
        H(X) = 0.5\log 2 + 0.25\log 4 + 0.125\log 8 + 0.125\log 8 = 1.75 \text{ bits}
    \end{equation*}
    The expected length of this code is also $1.75$:
    \begin{equation*}
        L(C) = 0.5 \cdot 1 + 0.25 \cdot 2 + 0.125 \cdot 3 + 0.125 \cdot 3 = 1.75 \text{ bits}
    \end{equation*}
    In this example we have seen a code that is optimal in the sense that the expected length of the code is equal to the entropy of the random variable.
\end{example}

\begin{example}[Morse Code]\label{ex:morse_code}
    bla bla from \cite{ElementsofInformationTheory}
\end{example}

\begin{definition}[Nonsingular Code]\label{def:nonsingular_code}
    bla bla from \cite{ElementsofInformationTheory}
\end{definition}

\begin{definition}[Extension of a Code]\label{def:extension_code}
    bla bla from \cite{ElementsofInformationTheory}
\end{definition}

\begin{definition}[Unique Decodability]\label{def:unique_decodability}
    bla bla from \cite{ElementsofInformationTheory}
\end{definition}

\begin{definition}[Prefix Code]\label{def:prefix_code}
    bla bla from \cite{ElementsofInformationTheory}
\end{definition}

\subsection{Kraft's Inequality}

Some introduction from \cite{ElementsofInformationTheory}

\begin{theorem}[Kraft's Inequality]\label{thm:kraft_inequality}
    bla bla from \cite{ElementsofInformationTheory}
\end{theorem}
\begin{proof}
    bla bla from \cite{ElementsofInformationTheory}
\end{proof}

\subsection{Source Coding Theorem}

Some introduction from \cite{ElementsofInformationTheory,Shannon1948,KolmogorovComplexity,han2002mathematics}

\begin{theorem}[Source Coding Theorem]\label{thm:source_coding_theorem}
    bla bla from \cite{ElementsofInformationTheory,han2002mathematics}
\end{theorem}
\begin{proof}
    bla bla from \cite{ElementsofInformationTheory,han2002mathematics}
\end{proof}







\clearpage
\section{Integer Coding}

Introduction to integer coding, and some theorems on unary coding. Almost every thing from \cite{ferragina2023pearls} and \cite{han2002mathematics}

\subsection{Elias Codes: $\gamma$ and $\delta$}
\begin{itemize}
    \item Definition of Elias codes
    \item Theorems on space occupancy
\end{itemize}

\subsection{Rice Code}
Just a brief mention
\subsection{Elias-Fano Code}
All form \cite{ferragina2023pearls,sayood2002lossless}
\begin{itemize}
    \item Definition of Elias-Fano code and examples
    \item Theorem on the space bound of Elias-Fano code
    \item Introduction to \texttt{Access(i)} and \texttt{NextGEQ(x)}
\end{itemize}

\section{Statistical Coding}

Introduction to statistical coding, from \cite{han2002mathematics} and \cite{ferragina2023pearls}

\subsection{Huffman Coding}

\subsection{Arithmetic Coding}

\section{Higher Order Entropy}

\section{Bitvectors}

Introduction to bitvectors, from \cite{ferragina2023pearls} and \cite{navarro2016compact}

\subsection{Access}
\subsection{Rank}
\subsection{Select}
\subsection{RRR: A Space-Efficient Rank/Select Structure}
