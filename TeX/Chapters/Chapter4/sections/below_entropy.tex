\section{Achieving Sub-Entropy Space for Path Queries}
\label{sec:below_entropy}

To evaluate the space efficiency of our proposed structure, we establish a baseline based on the information content in the weighted DAG itself. Drawing upon the principles of information theory outlined in \autoref{ch:Chapter2}, we can define a measure of entropy for the graph $G=(V, E, w)$.

Any \emph{lossless} representation of the graph $G$ must, at a minimum, encode the information required to uniquely define its structure (the edges $E$) and the associated weights (the function $w$). We formulate a $0^{th}$-order entropy measure, denoted $\mathcal{H}_0(G)$, as a lower bound on the number of bits required to represent these components, assuming no prior knowledge about correlations or higher-order statistical properties.

The information content required to represent the sequence of vertex weights $(w(v))_{v \in V}$ constitutes the first component. A fundamental lower bound, denoted $\mathcal{H}_W(G)$, can be established by considering the minimal binary representation length for each individual weight:
\[ \mathcal{H}_W(G) = \sum_{v \in V} \lceil \log_2 (w(v)+1) \rceil \quad \text{bits}. \]
This measure reflects the space needed assuming each weight is encoded independently, using the minimal bits for its value, without leveraging potential statistical correlations or distribution patterns suitable for techniques like variable-length integer coding (\autoref{sec:integer_coding}).

The second component relates to the graph's topology, specifically the set of edges $E$. With $n = |V|$ vertices, there exist $n(n-1)$ possible directed edges (excluding self-loops). Encoding the topology requires identifying which $m = |E|$ of these potential edges are present. Assuming all directed graphs with $n$ vertices and $m$ edges are equally probable a priori, the information content is determined by the number of ways to choose these $m$ edges. This leads to the topological information component, $\mathcal{H}_E(G)$:
\[ \mathcal{H}_E(G) = \log_2 \binom{n(n-1)}{m} \quad \text{bits}. \]
This quantity can be approximated using Stirling's formula,
\[\log_2 \binom{N}{k} \approx k \log_2(N/k) + O(k)\]
yielding
\[\mathcal{H}_E(G) \approx m \log_2 \left( \frac{n(n-1)}{m} \right) + O(m) \quad \text{bits}\]

Combining these components gives us a formal definition for the $0^{th}$-order entropy of the weighted DAG.

\begin{definition}[$0^{th}$-Order Weighted DAG Entropy]
    \label{def:dag_entropy}
    For a weighted DAG $G = (V, E, w)$ with $n=|V|$ vertices and $m=|E|$ edges, its $0^{th}$-order entropy $\mathcal{H}_0(G)$ is defined as the sum of the information content required for the weights and the topology:
    \[ \mathcal{H}_0(G) = \sum_{v \in V} \lceil \log_2 (w(v)+1) \rceil + \log_2 \binom{n(n-1)}{m} \quad \text{bits}. \]
\end{definition}

The value $\mathcal{H}_0(G)$ represents a theoretical lower bound on the space required by any lossless encoding of the graph $(V, E, w)$ based solely on its zero-order statistics.

Our proposed succinct data structure (\autoref{sec:succinct_dag_representation}), however, is designed differently. While lossless for \Rank{} query computation (\ref{def:rank_dag}), it is inherently \emph{lossy} concerning the reconstruction of the original graph $G$, as it does not store the complete edge set $E$. It retains only vertex weights $\mathcal{W}$, successor information $\Sigma$, and associated data $\mathcal{D}$. This distinction allows the structure's space usage, $S(G)$, to potentially fall below the $\mathcal{H}_0(G)$ bound.

To illustrate this, we analyze performance on a weighted DAG derived from unrolling a Bitcoin Peer-to-Peer temporal network graph \cite{kumar2018rev2} (details of the unrolling process are beyond the scope of this thesis), having $n = 22,210$ vertices and $m = 50,514$ edges. For this specific DAG, the calculated $0^{th}$-order entropy $\mathcal{H}_0(G)$ amounts to 1,525,730 bits, comprising $\mathcal{H}_W(G) = 60,824$ bits for weights and $\mathcal{H}_E(G) = 1,464,906$ bits for topology according to \ref{def:dag_entropy}.

We compare this theoretical lower bound $\mathcal{H}_0(G)$ against theoretical estimates of the space required by our succinct structure and alternative approaches based on precomputation.

For sequences composed of general non-negative integers $x$, such as the vertex weights $\mathcal{W}$, the successor identifiers $\Sigma$, or the interval endpoints in baseline precomputation results, the space estimation is based on summing the minimal binary representation cost for each integer independently. This cost is calculated as $\lceil \log_2(x+1) \rceil$ bits per integer $x$.

When representing strictly monotonic sequences, like the run start values in RLE or interval endpoints compressed using Elias-Fano (\autoref{sec:elias_fano_code}), our space estimation relies on its established theoretical complexity. \autoref{thm:ef_space} guarantees an upper bound of
\[n \log_2(u/n) + 2n\]
bits for encoding $n$ integers within the range $[0, u)$ using Elias-Fano.

Finally, the space for the offset sequences $\mathcal{I}_v$ when stored using the Run-Length Encoding (RLE) scheme described in \autoref{sec:compression_strategies} is estimated by combining the previous principles. It requires the sum of the space estimate for the strictly monotonic sequence of run start values (using the Elias-Fano estimation) and the space estimate for the sequence of corresponding run lengths (using the minimal binary representation cost for each length).

\begin{table}[htbp]
    \centering
    \small
    \begin{tabular}{l r}
        \toprule
        Component / Method                                      & Estimated Bits \\
        \midrule
        \textbf{Theoretical Lower Bound ($\mathcal{H}_0(G)$)}   &                \\
        \quad $0^{th}$-Order Entropy Total                      & 1,525,730      \\
        \quad \quad Weights Component ($\mathcal{H}_W(G)$)      & 60,824         \\
        \quad \quad Topology Component ($\mathcal{H}_E(G)$)     & 1,464,906      \\
        \midrule
        \textbf{Precomputed Rank Queries}                       &                \\
        \quad Explicit Storage (Minimal Binary)                 & 4,854,533      \\
        \quad Elias-Fano Compressed Storage                     & 2,211,849      \\
        \midrule
        \textbf{Succinct DAG Representation (RLE)}              &                \\
        \quad Total Space ($S(G)$)                              & 602,808        \\
        \quad \quad Weights $\mathcal{W}$ (Minimal Binary)      & 60,824         \\
        \quad \quad Successors $\Sigma$ (Minimal Binary)        & 297,700        \\
        \quad \quad Associated Data $\mathcal{D}$ (RLE Offsets) & 244,284        \\
        \bottomrule
    \end{tabular}
    \caption{Theoretical space estimates (in bits) for the example Bitcoin DAG ($n=22,210$, $m=50,514$).}
    \label{tab:space_estimates_bitcoin}
\end{table}

Table \ref{tab:space_estimates_bitcoin} presents the results of this theoretical space estimation for the example DAG. The total estimated space for our succinct DAG representation using RLE for the offsets, $S(G)$, is 602,808 bits. This value is notably less than half of the $0^{th}$-order graph entropy $\mathcal{H}_0(G)$ (1,525,730 bits), demonstrating the benefit of storing only query-relevant information rather than the full graph topology.

Furthermore, the comparison against precomputation strategies underscores the practical advantages of our approach. Precomputing and storing all \Rank{} query results explicitly represents a baseline for achieving minimal query time (potentially $O(1)$ access), but at a significant space cost, estimated at 4,854,533 bits even with minimal binary encoding. Attempting to mitigate this by compressing the precomputed results using Elias-Fano still requires 2,211,849 bits. While this compression offers substantial savings over the explicit form, the resulting space remains considerably larger than both the graph entropy bound $\mathcal{H}_0(G)$ and, more importantly, the space achieved by our succinct structure ($S(G)$).

Therefore, our analysis indicates that the proposed succinct DAG representation provides not only a mechanism to answer complex path queries but does so with remarkable space efficiency. It avoids the prohibitive space overhead associated with direct precomputation strategies, offering a compact alternative that surpasses even optimized precomputation methods in terms of storage footprint, and falls below conventional entropy bounds for lossless graph representation due to its targeted, query-specific information retention.
