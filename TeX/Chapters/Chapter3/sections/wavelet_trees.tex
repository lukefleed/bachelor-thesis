\clearpage
\section{Wavelet Trees}
\todo[fancyline]{Improve this introduction, just a draft}
\noindent Wavelet trees, introduced in 2003 by Grossi, Gupta, and Vitter \cite{GrossiWT2003,} are a self indexing data structure: meaning they can answer rank and select queries, while still allowing to access the text. This combination makes them particularly useful for compressed full-text indexes like the FM-index \cite{ferragina2000opportunistic}. In such indexes, wavelet trees are employed to efficiently answer rank queries during the search process. \vspace{0.4cm}

% Its name reflects its similarity to wavelet packet decomposition, where the sequence's "high" and "low" symbol values are separated and recursively subdivided. \vspace{0.4cm}

\todo[inline, color=blue!30]{Upon closer examination, one can recognize that the wavelet tree is a slight extension of an older (1988) data structure by Chazelle \cite{Chazelle1988}, commonly used in Computational Geometry. This structure represents points on a two-dimensional grid, undergoing a reshuffling process to sort them by one coordinate and then by the other. Kärkkäinen (1999) \cite{karkkainen1999repetition} was the first to apply this structure to text indexing, although the concept and usage differed from Grossi et al.'s proposal four years later} \vspace{0.4cm}

\noindent Wavelet Trees can be seen in different ways: (i) as sequence representation, (ii) as a permutation of elements, and (iii) as grid point representation. Since 2003, these perspectives and their interconnections have proven valuable across diverse problem domains, extending beyond text indexing and computational geometry, where the structure originated \cite{WTForALL,WTFromTheoryToPractice,TheMyriadVirtuesWT}.

\subsection*{An introduction to the problem}
Consider a sequence $S[1,n]$ as a generalization of bitvectors whose elements $S[i]$ are drawn from an alphabet $\Sigma$\footnote{The size of the alphabet varies depending on the application. For example, in DNA sequences, the alphabet is $\Sigma = \{A,C,G,T\}$ (in \autoref{ch:Chapter4} we will focus more on this specific case), while in other case it could be of millions of characters, such as in natural language processing.}. We are interested in the following operations on the sequence $S$:
\begin{itemize}
    \item \texttt{Access(i)}: return the $i$-th element of $S$.
    \item \texttt{Rank(c,i)}: return the number of occurrences of character $c$ in the prefix $S[1,i]$.
    \item \texttt{Select(c,i)}: return the position of the $i$-th occurrence of character $c$ in $S$.
\end{itemize}
However, dealing with sequences is much more complex than dealing with bitvectors (as we have seen in \autoref{sec:bitvectors}). In \cite{navarro2016compact} shows how a naive approach to solve this problem would require $n\sigma + o(n\sigma)$ bits of space, which is not space-efficient. Consider $\sigma$ bitvectors of length $n$, one for each symbol in the alphabet such that the $i$-th bit of the $c$-th bitvector is 1 if $S[i] = c$ and 0 otherwise. Then answering a $rank$ and $select$ query would be done by this simple transformation
\begin{align*}
    rank_c(S,i)   & = rank_1(B_c,i)   \\
    select_c(S,j) & = select_1(B_c,j)
\end{align*}
If we try to use the techniques from \autoref{sec:bitvectors} to compress the bitvectors, we would end up with a constant time complexity for the $rank$ and $select$ queries, but with the downside of a space occupancy of $n\sigma + o(n\sigma)$ bits. This is not space-efficient considering that the plain representation of the string requires $n\log \sigma + o(n)$ bits.\footnote{Even if we use a compressed representation of the bitvectors, the space occupancy would still have the dominant term $n\sigma$, that is at least $\Omega(n\sigma \log \log n / \log n)$ bits if we still want to support constant time $rank$ and $select$ queries.}
\begin{remark}[Notation]
    From now on, let $S[1,n]$ = $s_1s_2\dots s_n$ be a sequence of length $n$ over an alphabet $\Sigma$ that for simplicity we write as $\Sigma = \{1,\dots,\sigma\}$. In this way, the string can be represented using $n \lceil \log \sigma \rceil = n \log \sigma + o(n)$ bits in plain form.
\end{remark}

\subsection{Structure}

In the beginning of this section we showed that storing one bitvector per symbol is not space-efficient. The wavelet tree is a data structure that solves this problem by using a recursive hierarchical partitioning of the alphabet. Consider the subset $[a,b] \subset [1, \dots, \sigma]$, then a wavelet tree over $[a,b]$ is a balanced binary tree with $b-a+1$ leaves\footnote{if $a=b$ then the tree is just a leaf}. The root node $v_{root}$ is associated with the whole sequence $S[1,n]$, and stores a bitmap $B_{v_{root}}[1,n]$ defined as follows: $B_{v_{root}}[i] = 0$ if $S[i] \leq (a+b)/2$ and $B_{v_{root}}[i] = 1$ otherwise. The tree is then recursively built by associating the subsequence $S_0[1,n_0]$ of elements in $[a, \dots,\lfloor (a+b)/2 \rfloor ]$ to the left child of $v$, and the subsequence $S_1[1,n_1]$ of elements in $[\lfloor (a+b)/2 \rfloor +1,\dots, b]$ to the right child of $v$. This process is repeated until the leaves are reached. In this way the left child of the root node, is a wavelet tree for $S_0[1,n_0]$ over the alphabet $[a,\dots, \lfloor (a+b)/2 \rfloor ]$, and the right child is a wavelet tree for $S_1[1,n_1]$ over the alphabet $[\lfloor (a+b)/2 \rfloor +1,\dots, b]$. \cite{WTForALL} \vspace{0.4cm}

\noindent Building a wavelet tree is a recursive process that takes $O(n\log \sigma$) time by processing each node of the tree in linear time. The steps are outlined in Algorithm \ref{alg:build_wt}. Excluding the sequence $S$ and the final wavelet tree $T$, the algorithm uses $n \log \sigma$ bits of space \footnote{While building the wavelet tree, we can store the sequence $S$ on disk to free memory.}. \vspace{0.4cm}
\begin{example}[Wavelet Tree]
    TODO: Add an example of a wavelet tree for a sequence with small alphabet
\end{example}

\begin{algorithm}[h!]
    \caption{Building a wavelet tree}\label{alg:build_wt}
    \begin{algorithmic}
        \Require Sequence $S[1,n]$ over alphabet $\Sigma = \{1,\dots,\sigma\}$
        \Ensure Wavelet tree $T$ for $S$
        \vspace{0.4cm}

        \Function{build\_wt}{$S,n$}
        \State $T \gets build(S,n,1,\sigma)$
        \State \Return $T$
        \EndFunction

        \vspace{0.4cm}
        \Function {build}{$S,n,a,b$} \Comment{\small{Takes a string $S[1,n]$ over $[a,b]$}}
        \If {$a = b$}
        \State Free S
        \State \Return null
        \EndIf
        \State $v \gets$ new node
        \State $m \gets \lfloor (a+b)/2 \rfloor$
        \State $z \gets 0$ \Comment{\small{number of elements in $S$ that are $\leq m$}}
        \For {$i \gets 1$ to $n$}
        \If {$S[i] \leq m$}
        \State $z \gets z+1$
        \EndIf
        \EndFor

        \State Allocate strings $S_{left}[1,z]$ and $S_{right}[1,n-z]$
        \State Allocate bitmap $v.B[1,n]$
        \State $z \gets 0$

        \For {$i \gets 1$ to $n$}
        \If {$S[i] \leq m$}
        \State \texttt{bitclear}($v.B,i$) \Comment{\small{set $i$-th bit of $v.B$ to 0}}
        \State $z \gets z+1$
        \State $S_{left}[z] \gets S[i]$
        \Else
        \State \texttt{bitset}($v.B,i$) \Comment{\small{set $i$-th bit of $v.B$ to 1}}
        \State $S_{right}[i-z] \gets S[i]$
        \EndIf
        \EndFor

        \State Free S
        \State $v.left \gets \texttt{build}(S_{left},z,a,m)$
        \State $v.right \gets \texttt{build}(S_{right},n-z,m+1,b)$
        \State Pre-process $v.B$ for $rank$ and $select$ queries
        \State \Return $v$
        \EndFunction
    \end{algorithmic}
\end{algorithm}


\begin{remark} \label{rem:space_occupancy_wavelet_tree}
    The wavelet tree described has $\sigma$ leaves and $\sigma-1$ internal nodes, and the height of the tree is $ \lceil \log \sigma \rceil$. The space occupancy of each level it's exactly $n$ bits, while we have at most $n$ bits for the last level. The total number of bits stored by the wavelet tree in then upper bounded by $n \lceil \log \sigma \rceil$ bits. \cite{WTForALL}. However, if we also interested in storing the topology of the wavelet tree, then another $O(\sigma \log n)$, that can be critical for large alphabets. \cite{WTForALL}
\end{remark}

\subsubsection*{Tracking symbols}
We have seen how the wavelet tree serves as a representation for a string $S$, but more than that it is a succinct data structure for the string. Thus, it takes space asymptotically close to the plain representation of the string and allows us to access the $i$-th symbol of the string in $O(\log \sigma)$ time. In algorithm \ref{alg:access_wt} we show how extract the $i$-th symbol of the string $S$ using a wavelet tree $T$, this operation is called \texttt{Access}. \vspace{0.4cm}

\begin{algorithm}[h!]
    \caption{\texttt{Access} queries on a wavelet tree}\label{alg:access_wt}
    \begin{algorithmic}
        \Function {access}{$T,i$} \Comment{\small{$T$ is the sequence $S$ seen as a wavelet tree}}
        \State $v \gets T_{root}$ \Comment{\small{start at the root node}}
        \State $[a,b] \gets [1,\sigma]$
        \While{$a \neq b$}
        \If{$access(v.B,i) =0$} \Comment{\small{$i$-th bit of the bitmap of $v$}}
        \State $i \gets rank_0(v.B,i)$
        \State $v \gets v.left$ \Comment{\small{move to the left child of node $v$}}
        \State $b \gets \lfloor (a+b)/2 \rfloor$
        \Else
        \State $i \gets rank_1(v.B,i)$
        \State $v \gets v.right$ \Comment{\small{move to the right child of node $v$}}
        \State $a \gets \lfloor (a+b)/2 \rfloor +1$
        \EndIf
        \EndWhile
        \State \Return $a$
        \EndFunction
    \end{algorithmic}
\end{algorithm}

\noindent In order to find $S[i]$, we first look at the bitmap associated with the root node of the wavelet tree, and depending on the value of the $i$-th bit of the bitmap, we move to the left or right child of the root node and continue recursively. However, the problem is to determine where our $i$ has been mapped to: if we move to the left child, then we need to find the $i$-th 0 in the bitmap of the left child, and if we move to the right child, then we need to find the $i$-th 1 in the bitmap of the right child. This is done by the $rank_0$ and $rank_1$ functions, respectively. We continue this process until we reach a leaf node, and then we return the value of the leaf node.

\begin{example}
    TODO and TBD: add a figure with the wavelet tree and the process of finding the $i$-th symbol of the string.
\end{example}

\noindent In addition to retrieving the $i$-th symbol of the string, we might also need to perform the inverse operation. That is, given a symbol's position at a leaf node, we aim to determine the position of the symbol in the string. This operation is referred to as \texttt{Select} and is outlined in Algorithm \ref{alg:select_wt}. Assume we start at a given leaf node $v$ and want to find the position of the $j$-th occurrence of symbol $c$ in the string. We recursively move to the left or right child of the node $v$: if the leaf is the right child of its parent, then we need to find the $j$-th 1 in the bitmap of the parent node, and if the leaf is the left child of its parent, then we need to find the $j$-th 0 in the bitmap of the parent node. This is done by the $select_0$ and $select_1$ functions, respectively. We continue this process until we reach the root node, and then we return the position of the symbol in the string. As we have seen in \autoref{sec:bitvectors}, this two single operations can be solved in constant time using $n$ bits of $B$ and $o(n)$ bits of additional space. Thus, the time complexly to perform a \texttt{Select} query on a wavelet tree is $O(\log \sigma)$.

\begin{algorithm}
    \caption{\texttt{Select} queries on a wavelet tree}\label{alg:select_wt}
    \begin{algorithmic}
        % add two functions to the algorithmic environment
        \Function{$\text{select}_c$}{$S,j$}
        % \Require Sequence $S$ (as a wavelet tree $T$), symbol $c$, position $j$
        % \Ensure The position of the $j$-th occurrence of $c$ in $S$

        \State \Return $\text{select}(T._{root},1, \sigma, c, j)$
        \EndFunction

        \vspace{0.4cm}

        \Function{$\text{select}$}{$v,a,b,c,j$}
        % \Comment $v$ is the current node, $[a,b]$ is the alphabet range, $c$ is the symbol, $j$ is the position
        \If{$a = b$}
        \State \Return $j$
        \EndIf

        \If {$c \leq \lfloor (a+b)/2 \rfloor$}
        \State $j$ $\gets$ $\text{select}(v.left, a, \lfloor (a+b)/2 \rfloor, c, j)$
        \Return $select_0(v.B,j)$

        \Else
        \State $j$ $\gets$ $\text{select}(v.right, \lfloor (a+b)/2 \rfloor +1, b, c, j)$
        \State \Return $select_1(v.B,j)$
        \EndIf

        \EndFunction

    \end{algorithmic}
\end{algorithm}

\noindent During the $select$ algorithm, we track upwards the path from the leaf to the root. The process for solving a $rank$ query is similar, but instead of moving from the leaf to the root, we move from the root to the leaf. Algorithm \ref{alg:access_wt} also gives us the number of occurrences of a symbol $S[i]$ in the prefix $S[1,i]$, i.e $rank_{S[i]}(S,i)$. We now want to generalize this operation to solve any rank query $rank_c(S,i)$, where $c$ is a symbol in the alphabet. This is done by the algorithm \ref{alg:rank_wt}. The time complexity to perform a \texttt{Rank} query on a wavelet tree is $O(\log \sigma)$, just as for the \texttt{Select} query and the \texttt{Access} query.


\begin{algorithm}[h!]
    \caption{\texttt{Rank} queries on a wavelet tree}\label{alg:rank_wt}
    \begin{algorithmic}
        \Function{$\text{rank}_c$}{$S,i$}
        \State $v \gets T_{root}$ \Comment{\small{start at the root node}}
        \State $[a,b] \gets [1,\sigma]$
        \While {$a \neq b$}
        \If {$c \leq \lfloor (a+b)/2 \rfloor$}
        \State $i \gets rank_0(v.B,i)$
        \State $v \gets v.left$ \Comment{\small{move to the left child of node $v$}}
        \State $b \gets \lfloor (a+b)/2 \rfloor$
        \Else
        \State $i \gets rank_1(v.B,i)$
        \State $v \gets v.right$ \Comment{\small{move to the right child of node $v$}}
        \State $a \gets \lfloor (a+b)/2 \rfloor +1$
        \EndIf
        \EndWhile
        \State \Return $i$
        \EndFunction
    \end{algorithmic}

\end{algorithm}

% \noindent TODO: Add some pratical consideration about using constant-time $rank$ and $select$ on the bitvectors. \cite{navarro2016compact} 6.2

\todo[inline, color=blue!30]{
    TBD if to add: In \ref{rem:space_occupancy_wavelet_tree} we mentioned that storing the topology of the wavelet tree requires $O(\sigma \log n)$ bits. This may be critical for large alphabets, and in this section we will show that this term can be removed by slightly altering the balanced wavelet tree shape. \cite{MAKINEN2007332,MAKINEN2006703}. \vspace{0.4cm}

    Look for section 2.3 of \cite{WTForALL}
}
