% Chapter 1

\chapter{Introduction} % Chapter title

\label{ch:introduction} % For referencing the chapter elsewhere, use \autoref{ch:introduction}

%----------------------------------------------------------------------------------------


\marginpar{A well-balanced line width improves the legibility of the text. That's what typography is all about, right?} So far, many theses, some books, and several other publications have been typeset successfully with it. If you are interested in some typographic details behind it, enjoy Robert Bringhurst's wonderful book. % \citep{bringhurst:2002}.

\paragraph{Important Note:} Some things of this style might look unusual at first glance, many people feel so in the beginning. However, all things are intentionally designed to be as they are, especially these:
\begin{itemize}
\item No bold fonts are used. Italics or spaced small caps do the job quite well.
\item The size of the text body is intentionally shaped like it is. It supports both legibility and allows a reasonable amount of information to be on a page. And, no: the lines are not too short.
\item The tables intentionally do not use vertical or double rules. See the documentation for the \texttt{booktabs} package for a nice discussion of this topic.\footnote{To be found online at \\ \url{http://www.ctan.org/tex-archive/macros/latex/contrib/booktabs/}.}
\item And last but not least, to provide the reader with a way easier access to page numbers in the table of contents, the page numbers are right behind the titles. Yes, they are \emph{not} neatly aligned at the right side and they are \emph{not} connected with dots that help the eye to bridge a distance that is not necessary. If you are still not convinced: is your reader interested in the page number or does she want to sum the numbers up?
\end{itemize}
